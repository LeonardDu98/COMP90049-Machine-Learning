\documentclass[11pt]{article}
\usepackage{colacl}
\usepackage{multirow}
\sloppy



\title{Paper Template for COMP90049 Report}
\author
{Anonymous}



\begin{document}
\maketitle

\section{Introduction}

%引用nan jing
Genre classification and sentiment analysis are the two most mainstream application scenarios of movie classification technology. The sentiment analysis focus on movie review comments to get objective audiences' attitudes. Genre classification technology is used not only to judge the movie ratings to find the suitable age group but also to define movie categories, such as action, comedy, horror and so on. It's a very challenging classification problem that labelling movies according to their corresponding genre.

In this paper, the research process can be mainly divided into two steps: i) feature selection and representation, ii) model training and prediction. The former relies on term frequency–inverse document frequency (TF-IDF), while the latter bases on the Naive Bayes and the Support Vector Machine (SVM). The report shows the changes in the prediction results under different features or different models and further analyzes these changes.

%This is a report template, suitable for \LaTeX (it doesn't model very much).
%Don't use fonts smaller than this one (Times 11). Don't include a title page,
%table of contents, abstract, or other similar front matter.

%Please don't include your name and/or student ID in the title or header; 
%your report should be anonymised for the reviewing process. Don't forget
%that you must cite the corpus \cite{WikiMisspell}.

%GC和SA是电影分类技术最主流的两个应用场景。SA技术通过分析影评来得到客观的用户观影体验。GA技术不仅可以用于电源评级判断适合的观影人群,而且可以定义电影类别,如动作,喜剧,恐怖电影等等。It's a very challenging classification problem that labeling movies according to their corresponding genre.

%在这篇文章中,研究过程主要可以分为两步:特征选择和模型训练预测。前者依赖于TFIDF分类,后者使用NBA和SVM做实现。文章展示了再不同特征或不同模型的情况下预测结果的变化,并对这些变化进行分析讨论。

\section{Related Work}

%Text,\footnote{Footnote text} with footnotes at bottom of page.
%大量引用
Over the years, researchers have proposed various automatic genre classification methods for text and general video data. Some methods rely on textual features, such as movie titles or summary user tags. Other methods rely on audio-visual features.

Rasheed et al. compute four video features: lighting key, motion content, average shot length and colour variance to predict movie genres. But they only divided movies into four categories: comedy, action, drama, and horror. Obviously, this broad classification method is no longer applicable. In addition, since certain categories are very similar visually such as documentary and comedy,using just visual features are not enough. In this scenario, the audio or text extracted from audio is necessary for genre classification. One hybrid method was proposed by Huang and Wang combining both low-level visual features and audio information.

Ho applied three methods to predict 10 most popular genres based on the dataset only containing 16,000 movie titles. Overall, the SVM achieves the highest F1 score of 0.55 in his paper. Hoang explores Naive Bayes and Recurrent Neural Networks for text classification, and his model performs well in multi-label problems.

\section{Feature Selection}

\subsection{Data Set}

The data set used in this report is derived from a larger database script by Deldjoo et al. and F. Maxwell et al.

The data set is partitioned into three sets, a training set, a valid set and a test set. The training set and the valid set contain two files, the features data and the genres data. The training set is used to train and fit the model, and the valid set is used to test and analyze the differences between various models. The test set has only one file containing features data. We need to use a trained model to predict the genres of the test data. The prediction result is used for kaggle competition. Table 3.1 below shows the structure information of different files.

%The data set was partitioned into three sets, a training set, a valid set and a test set. 训练集和valid集合分别包含两个文件:电影的具体数据和电影的lable。训练集用于对模型进行训练拟合,valid集合用于测试分析各个模型之间的差别。测试集合只有一个包含电影数据的文件,我们需要使用训练好的模型对测试集合中的数据进行预测。预测结果用于kaggle的比赛。下表3.1展示了不同文件的结构信息。

\begin{table}[h]
 \begin{center}
\begin{tabular}{|l|l|l|}

      \hline
      Corpus & \multicolumn{2}{|c|}{Features}\\
      \hline\hline
      \multirow{2}{5.5em}{Training set} & train\_features & 5240 instances\\
      &train\_labels & 5240 instances\\
      \hline
      \multirow{2}{5.5em}{Valid set} & valid\_features & 299 instances\\
      &valid\_labels & 299 instances\\
      \hline
      Test set & test\_features & 235 instances\\
      \hline

\end{tabular}
\caption{The structure of the data}\label{table1}
 \end{center}
\end{table}

\subsection{Data Pre-processing}

For the special model, the input data should have the same structure. Therefore, we should perform the same operation on train\_features, valid\_features and test\_features. Below I will only explain the processing of train\_features.

Preprocessing the original data is necessary and effective to promote the training of the model, because there are NaN and invalid data in train\_features. Abnormal data can be divided into three categories:

Since "MovieId" and "YTId" do not contain any information related to the movie category, eliminating them from the training set does not affect the accuracy of the model

The "year" represents the release time of the movie. But the data of "year" in some instances is missed. Compared with "title" and "tags", the "year" contains very little movie information, although movies of similar types may be released at the same time. I choose to exclude the "year" column.

In all instances, the values of "avf31", "avf32" and "avf104" are the same. I have reliable reasons to judge that these three data do not represent any characteristics of the movies. Therefore, I can eliminate these three columns.

%同一模型,要求有确定的变量格式。所以我对vaild_Features和test_Features也做上述操作。满足数据格式的一致性。

%train_Features中的原始数据中存在NaN或者无效数据。对原始数据进行预处理有效的方便之后模型的训练。本次文件中的需要处理的数据可以分为三种。

%"movieId" 和 “YTId” 并不包含任何和电影内容相关的信息,这两个数据只会增加训练的难度,所以我把它们从训练数据中剔除。

%“year”代表电影的发行时间。但是数据集合中部分例子的year信息缺失。和“title”,“tags”相比,year所包含的电影信息极少,虽然同一时间相似类型的电影可能会大量发行。我选择剔除“year”这一列。
%通过筛查,我发现'avf31', 'avf32', 'avf104'这三个features在所有的instances中的值都是相同的。 我有理由判断这三个数据并不能代表任何instances的特征。所以我选择剔除这三列。

\subsection{Term Frequency–Inverse Document Frequency}

Since the "title" and "tags" are textual data, We cannot directly know which word is important. Term Frequency–Inverse Document Frequency (TF-IDF) is a numerical statistic that is intended to reflect how important a word is to a document in the corpus.

Calculate the TF at first. The weight of a term that occurs in a document is simply proportional to the term frequency. In other words, in one instance, the more times a word appears, the more important the word. Therefore we can define:
\[tf(t, d) =  f_{t,d}\]
$f_{t,d}$: the number of times that term $t$ occurs in document $d$

Then, we need calculate the IDF. The specificity of a term can be quantified as an inverse function of the number of documents in which it occurs. In other words, A word is not important if it appears in most examples. So we can define:
\[idf(t, D) = log(D/n_t)\]
$D$: the number of instances
$n_t$: the number of instances containing $t$.

Finally, we can get the tf-idf:
\[tf-idf = tf(t, d) * idf(t, D)\]
The greater the tf-idf, the more important the word.

\subsection{Standardization or Normalization}

Comparing different columns, we can find that the values between the columns are very different. For example, "avf1" is generally less than 0.5, but "avf15" is generally greater than 10. If we directly train this data, when "avf1" and "avf15" have the same coefficient, the impact of "avf15" will be much larger than that of "avf1". However, the data of different columns should have the same weight. Therefore, we need to standardize or normalize data.

By calculating the value of the Gaussian error function for each column, we can find that most data does not conform to the normal distribution. For example, "avf10" has only four discrete values. If we use standardization, the error of the training result may increase. Therefore, I use normalization to process the data, and the range of each column is [0,1].

\[X^{'} = (X - X_{min}) / (X_{max} - X_{min})\]

\section{Methodology}

Because this is a classification problem, in a large number of machine learning models, we should first consider the classification algorithm,such as Naive Bayes, SVM, K-nearest neighbours and Decision tree. I verified these four models in the code and found that Naive Bayes and SVM performed better. This article will focus on the in-depth analysis of Naive Bayes and SVM.

\subsection{Naive Bayes}
%%饮用
The naive Bayes method is based on the conditional independence assumption with the decision rule, $maximum a posteriori$ (MAP), to classify an instance. It is a mainstream spam classification algorithm and has great advantages in processing textual data. In the original data, both the $title$ and the $lag$ are textual data, and naive Bayes may have an ideal training effect.

Naive Bayes can be roughly divided into three categories: Gaussian naive Bayes, Multinomial naive Bayes, and Bernoulli naive Bayes. When dealing with continuous data, one common assumption is the data is distributed according to a normal distribution. Therefore, we cannot calculate the exact probability of a giving point since the data is not discrete. However, we can get the mean and variance of the sample to calculate the density function of normal distribution. Using the density function to replace the probability is the kernel of Gaussian naive Bayes.

When the data is discrete and finite, samples (feature vectors) represent the frequencies with which certain events have been generated by a multinomial. Multinomial naive Bayes is an event model commonly used for document classification, where events represent the occurrence of words in a single document.

Like the multinomial model, the Bernoulli model is also suitable for discrete features. The difference is that the value of each element in the Bernoulli model can only be 1 and 0. If we only need to judge whether the keyword appears, Bernoulli naive Bayes is the best model.

\subsection{Support Vector Machine (SVM)}
%%饮用
SVM is supervised learning models with associated learning algorithms. This method constructs a maximum-margin hyperplane which aims to lower the generalisation error of the classification. It can be used for both regression and classification tasks. Usually, it is widely used in classification labels.

With the help of kernel function, SVM can handle both linear and non-linear data classification. The most important theory is mapping the feature vectors in low-dimensional space to high-dimensional space where we can easiler to find linear separability. Besides, when we apply the kernel trick to maximum-margin hyperplanes, we can get a nonlinear classifier with curved borders.

\section{Results}

\subsection{Different TFIDF parameters}

The Multinomial naive Bayes model is used here to determine the effect of different TF-IDF parameters on the accuracy of the model.
Above all, stop words have no meaning, so we use $stop\_words='english'$ to eliminate all stop words. Besides, there are two important parameters: $min\_df$ and $max\_df$.

$min\_df$ is a cut-off parameter since every word in word bag must be contained in at least $min\_df$ instances in the whole data set. If float in the range of [0.0, 1.0], the parameter represents a proportion of documents。It can effectively avoid the impact of extreme data that has only appeared once or twice on the model. $max\_df$ have the opposite effect that can delete the common words.

There are more than 5,000 data in the "title" words bag. Without filtering, the difficulty of model training will greatly increase. At the same time, too many features may cause the curse of dimensionality. However, the "tag" words bag only have 206 words. It's easy to handle hundreds of features. Therefore, we can use $max\_df = 0.3$ to delete the most common words in "title" and "tag".

$max\_df$ is 0.3, $min\_df$ expands from 1 to 9, the accuracy rate increases from 24\% to 38\%. The word bag size is reduced to 149 from 5819. When $min\_df$ expands from 9 to 20, the accuracy rate drop to 36\%。Since then, continue to expand the $min\_df$, the accuracy rate has been maintained at around 36\%.

we can have a conclusion that when $min\_df$ is very small, the samples in the training set are very large, and our model cannot effectively learn the information from data. As the training set decreases, the accuracy rate increases steadily. However, when $min\_df$ is greater than 9, the training sample becomes smaller, and the accuracy rate begins to gradually decline, indicating that the features cannot effectively reflect the information of "title". When $min\_df$ is greater than 20, the data extracted from "title" has become insignificant and has no significant effect on the model. We can make bold predictions, 36\% accuracy rate is the model ability to predict without "title" columns.

\subsection{Different Naive Bayes models}
Now we stipulate $min\_df=9$ and $max\_df = 0.3$. Let us observe the effect of different Bayes models on accuracy. Before testing, I predict that the Multinomial Naive Bayes will perform best, while the Gaussian Naive Bayes and Bernoulli Naive Bayes are very poor because most of our features are discrete data and do not satisfy the binomial distribution or Gaussian distribution.

Howeever, the accuracy of Multinomial Naive Bayes is 38.4\%, the accuracy of Bernoulli Naive Bayes is 31.6\% and the accuracy of Gaussian Naive Bayes is 7.6\%.

Why is Bernoulli Naive Bayes much better than Gaussian Naive Bayes in our project? I analyzed the training data. Many raw data are randomly distributed. However, the data extracted by tfidf about "title" and "label" usually only have two or three specific values. These data cannot fully satisfy the binomial distribution, but the data characteristics are very similar to the binomial distribution. When processing such data, Bernoulli Naive Bayes realizes much better than Gaussian Naive Bayes.

\subsection{Linear or Nonlinear SVM}

Now, I chose one popular nonlinear SVM, Radial basis function (RBF), to compare with sample linear SVM. The linear and RBF kernel are simply different in case of making the hyperplane decision boundary between the classes. Ususally, RBF can map the feature vectors in low-dimensional space to higher high-dimensional space than linear SVM.

The accuracy of Linear SVM is 39.2\%, but the accuracy of RBF SVM is 41.4\%. In my opinion, since RBF mapping the vectors to very high-dimensional, it's easiler to find separability than linear SVM. Therefore, it is more accurate. However, my computer has been running RBF SVM for more than 1 minute, and running B only took less than 15 seconds. Finally, the mapping process from low size to high size takes a lot of time. And because the RBF SVM data is more sparse, the risk of overfitting a single training set will increase.

\subsection{Naive Bayes vs SVM}

In this part, we just compare the Linear SVM with Multinomial Naive Bayes. The prediction accuracy of the two models is very close, and Linear SVM is only 0.8\% higher than Multinomial Naive Bayes.

I mentioned above Naive Bayes is a mainstream spam classification algorithm. This model usually performs well on textual data. However, SVM can handle both textual and numerical data. In our raw data, only a few columns are text, while more than 100 columns are numbers. Maybe it's the reason why SVM is better than Naive Bayes in this problem.

\section{Conclusions}

To summarize, this report evaluates the effectiveness of Naive Bayes and SVM on the classification of film genres. Based solely on the given data set, the overall accuracy of SVM is better than Naive Bayes. In addition, it is very important to modify the appropriate parameters for every model. Nowadays, more and more movies have more than one genre, and multi-label movie genre detection is the future improvement direction.


\bibliographystyle{acl}
\bibliography{sample}

\end{document}